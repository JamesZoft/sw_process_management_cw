\documentclass{article}
\usepackage{etoolbox} 
\usepackage{tocloft}
\usepackage{times}
\usepackage{graphicx}
\usepackage{url}
\usepackage[pdfborder={0 0 0}]{hyperref}
\usepackage[backend=bibtex, style=authoryear]{biblatex}
\usepackage[parfill]{parskip}
\bibliography{references}
\begin{document}

\title{CS4670 Coursework - Workout Recorder App}
\author{James Reed}
\maketitle
\newpage
\tableofcontents

\newpage
\renewcommand\thesection{\arabic{section}}

\section{Introduction}

The purpose of this report is to produce a full and complete set of requirements and goals for a workout recorder app, along with a detailed scope and diagrams to support all of this. I will first analyse what the client's reqirements are and how these translate into high level requirements, then break these down into functional and non-functional requirements. I will then use these requirements to produce a set of goals for the system and the needed diagrams (e.g. UML, sequence, collaboration, etc) that will further explore the system. The goal of this report will be such that, when it is finished, a developer or team of developers will be able to take it and build the system it describes.

\section{Requirements}

\subsection{High-level Requirements}

The high level requirements of this project are as follows:

\begin{itemize}

\item Record which workouts are done or gym classes are followed that day
\item Record progress made and goals achieved per workout
\item Be able to visualise achivements/goals via graphs or other statistical methods
\item App is able to upload progress, achievments and goals to a cloud-based service that the user can access
\item App is able to work without being connected to the internet
\item Non-technical and intuitive user interface
\item Development time of three to six months
\item Budget of \pounds20,000-\pounds30,000

\end{itemize}

\subsection{Functional Requirements}

The functional requirements for this system will be:

\begin{itemize}

\item User will be able to record the type and number of workouts/gym classes done that day and be able to accept workout types that the app doesn't list from the user via a form or similar
\item User should be able to record progress (time, distance, crunches done, weight lifted) and tick off goals (ran for 20 minutes, lifted 120lb)
\item Progress should, after being entered, immediately be available for comparison and dissemination by the user via statistics (graphs, tables etc)
\item Statistical analysis of progress, achievments and goals should be available for all workouts and exercise types, provided enough information is available
\item The user should be able to access their workout statistics from anywhere that is able to launch the app via a login system (other android/iOS devices, pc, etc)
\item The app should (if possible) upload the user's statistics to their user account 

\end{itemize}

\subsection{Non-functional Requirements}

The non-functional requirements for this system will be:

\begin{itemize}

\item 

\end{itemize}

\section{}

\newpage

\nocite{*}
\printbibliography[title={Citations}]

\end{document}

